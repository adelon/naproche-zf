\import{order/order.tex}
\import{relation.tex}
\import{set/suc.tex}
\import{ordinal.tex}


\section{The numbers}\label{form_sec_numbers}

\begin{signature}
    $\reals$ is a set.
\end{signature}

\begin{signature}
    $x + y$ is a set.
\end{signature}

\begin{signature}
    $x \rmul y$ is a set.
\end{signature}



\subsection{Creation of natural numbers}

\subsubsection{Defenition and axioms}

\begin{abbreviation}\label{inductive_set}
    $A$ is an inductive set iff $\emptyset\in A$ and for all $a\in A$ we have $\suc{a}\in A$.
\end{abbreviation}

\begin{abbreviation}\label{zero_is_emptyset}
    $\zero = \emptyset$.
\end{abbreviation}

\begin{axiom}\label{reals_axiom_zero_in_reals}
    $\zero \in \reals$.
\end{axiom}

\begin{axiom}\label{one_in_reals}
    $1 \in \reals$.
\end{axiom}

\begin{axiom}\label{zero_neq_one}
    $\zero \neq 1$.
\end{axiom}

\begin{axiom}\label{one_is_suc_zero}
    $\suc{\zero} = 1$.
\end{axiom}

\begin{axiom}\label{naturals_subseteq_reals}
    $\naturals \subseteq \reals$.
\end{axiom}

\begin{axiom}\label{naturals_inductive_set}
    $\naturals$ is an inductive set.
\end{axiom}

\begin{axiom}\label{naturals_smallest_inductive_set}
    Let $A$ be an inductive set.
    Then $\naturals\subseteq A$.
\end{axiom}

\begin{abbreviation}\label{is_a_natural_number}
    $n$ is a natural number iff $n \in \naturals$.
\end{abbreviation}

\begin{axiom}\label{naturals_addition_axiom_1}
    For all $n \in \naturals$ $n + \zero = \zero + n = n$.
\end{axiom}

\begin{axiom}\label{naturals_addition_axiom_2}
    For all $n, m \in \naturals$ $n + \suc{m} = \suc{n} + m = \suc{n+m}$.
\end{axiom}

\begin{axiom}\label{naturals_mul_axiom_1}
    For all $n \in \naturals$ we have $n \rmul \zero = \zero = \zero \rmul n$.
\end{axiom}

\begin{axiom}\label{naturals_mul_axiom_2}
    For all $n,m \in \naturals$ we have $\suc{n} \rmul m = (n \rmul m) + m$.
\end{axiom}

\begin{axiom}\label{addition_on_naturals}
    If $x$ is a natural number and $y$ is a natural number then $x+y$ is a natural number.
\end{axiom}

\begin{axiom}\label{naturals_rmul_is_closed_in_n}
    For all $n,m \in \naturals$ we have $(n \rmul m) \in \naturals$.
\end{axiom}

\begin{axiom}\label{naturals_add_is_closed_in_n}
    For all $n,m \in \naturals$ we have $(n + m) \in \naturals$.
\end{axiom}

\subsubsection{Natural numbers as ordinals}


\begin{lemma}\label{nat_is_successor_ordinal}
    Let $n\in\naturals$.
    Suppose $n\neq \emptyset$.
    Then $n$ is a successor ordinal.
\end{lemma}
\begin{proof}
    Let $M = \{ m\in \naturals \mid\text{$m = \emptyset$ or $m$ is a successor ordinal}\}$.
    $M$ is an inductive set by \cref{suc_ordinal,naturals_inductive_set,successor_ordinal,emptyset_is_ordinal}.
    Now $M\subseteq \naturals\subseteq M$
        by \cref{subseteq,naturals_smallest_inductive_set}.
    Thus $M = \naturals$.
    Follows by \cref{subseteq}.
\end{proof}

\begin{lemma}\label{nat_is_transitiveset}
    $\naturals$ is \in-transitive.
\end{lemma}
\begin{proof}
    Let $M = \{ m\in\naturals \mid \text{for all $n\in m$ we have $n\in\naturals$} \}$.
    $\emptyset\in M$.
    For all $n\in M$ we have $\suc{n}\in M$
        by \cref{naturals_inductive_set,suc}.
    Thus $M$ is an inductive set.
    Now $M\subseteq \naturals\subseteq M$
        by \cref{subseteq,naturals_smallest_inductive_set}.
    Hence $\naturals = M$.
\end{proof}

\begin{lemma}\label{natural_number_is_ordinal}
    Every natural number is an ordinal.
\end{lemma}
\begin{proof}
    Follows by \cref{suc_ordinal,suc_neq_emptyset,naturals_inductive_set,nat_is_successor_ordinal,successor_ordinal,suc_ordinal_implies_ordinal}.
\end{proof}

\begin{lemma}\label{omega_is_an_ordinal}
    $\naturals$ is an ordinal.
\end{lemma}
\begin{proof}
    Follows by \cref{natural_number_is_ordinal,transitive_set_of_ordinals_is_ordinal,nat_is_transitiveset}.
\end{proof}

\begin{lemma}\label{omega_is_a_limit_ordinal}
    $\naturals$ is a limit ordinal.
\end{lemma}
\begin{proof}
    $\emptyset\precedes \naturals$.
    If $n\in \naturals$, then $\suc{n}\in\naturals$.
\end{proof}


\subsubsection{Properties and Facts natural numbers}

\begin{theorem}\label{induction_principle}
    Let $P$ be a set.
    Suppose $P \subseteq \naturals$.
    Suppose $\zero \in P$.
    Suppose $\forall n \in P. \suc{n} \in P$.
    Then $P = \naturals$.
\end{theorem}
\begin{proof}
    Trivial.
\end{proof}

\begin{proposition}\label{existence_of_suc}
    Let $n \in \naturals$.
    Suppose $n \neq \zero$.
    Then there exist $n' \in \naturals$ such that $\suc{n'} = n$.
\end{proposition}
\begin{proof}
    Follows by \cref{transitiveset,nat_is_transitiveset,suc_intro_self,successor_ordinal,nat_is_successor_ordinal}.
\end{proof}

\begin{proposition}\label{suc_eq_plus_one}
    Let $n \in \naturals$.
    Then $\suc{n} = n + 1$.
\end{proposition}
\begin{proof}
    Let $P = \{ n \in \naturals \mid n + 1 = 1 + n  \}$.
    $\zero \in P$.
    It suffices to show that if $m \in P$ then $\suc{m} \in P$.
\end{proof}

\begin{proposition}\label{naturals_1_kommu}
    Let $n \in \naturals$.
    Then $1 + n = n + 1$.
\end{proposition}
\begin{proof}
    Follows by \cref{suc_eq_plus_one,naturals_addition_axiom_2,naturals_addition_axiom_1,naturals_inductive_set,one_is_suc_zero}.
\end{proof}

\begin{proposition}\label{naturals_add_kommu}
    For all $n \in \naturals$ we have for all $m\in \naturals$ we have $n + m = m + n$.
\end{proposition}
\begin{proof}
    Fix $n \in \naturals$.
    Let $P = \{ m \in \naturals \mid m + n = n + m \}$.
    It suffices to show that $P = \naturals$.
    $P \subseteq \naturals$.
    $\zero \in P$.
    It suffices to show that if $m \in P$ then $\suc{m} \in P$.
\end{proof}

\begin{proposition}\label{naturals_add_assoc}
    Suppose $n,m,k \in \naturals$.
    Then $n + (m + k) = (n + m) + k$.
\end{proposition}
\begin{proof}
    Let $P = \{ k \in \naturals \mid \text{for all $n',m' \in \naturals$ we have $n' + (m' + k) = (n' + m') + k$}\}$.
    $P \subseteq \naturals$.
    $\zero \in P$.
    It suffices to show that for all $k \in P$ we have $\suc{k} \in P$.
    Fix $k \in P$.
    It suffices to show that for all $n' \in \naturals$ we have for all $m' \in \naturals$ we have $n' + (m' + \suc{k}) = (n' + m') + \suc{k}$.
    Fix $n' \in \naturals$.
    Fix $m' \in \naturals$.
    \begin{align*}
        (n' + m') + \suc{k}    \\
        &= \suc{(n' + m') + k}   \\
        &= \suc{n' + (m' + k)} \\
        &= n' + \suc{(m' + k)} \\
        &= n' + (m' + \suc{k})
    \end{align*}
\end{proof}

\begin{proposition}\label{naturals_rmul_one_left}
    For all $n \in \naturals$ we have $1 \rmul n = n$.
\end{proposition}
\begin{proof}
    Fix $n \in \naturals$.
    \begin{align*}
        1 \rmul n  \\
        &= \suc{\zero} \rmul n \\
        &= (\zero \rmul n) + n \\
        &= (\zero) + n \\
        &= n
    \end{align*}
\end{proof}

\begin{proposition}\label{naturals_add_remove_brakets}
    Suppose $n,m,k \in \naturals$.
    Then $(n + m) + k = n + m + k = n + (m + k)$.    
\end{proposition}

\begin{proposition}\label{naturals_add_remove_brakets2}
    Suppose $n,m,k,l \in \naturals$.
    Then $(n + m) + (k + l)= n + m + k + l = n + (m + k) + l = (((n + m) + k) + l)$.    
\end{proposition}

%\begin{proposition}\label{natural_disstro_oneline}
%    Suppose $n,m,k \in \naturals$.
%    Then $n \rmul (m + k) = (n \rmul m) + (n \rmul k)$.
%\end{proposition}
%\begin{proof}
%    Let $P = \{n \in \naturals \mid \forall m,k \in \naturals .  n \rmul (m + k) = (n \rmul m) + (n \rmul k)\}$.
%    $\zero \in P$.
%    $P \subseteq \naturals$.
%    It suffices to show that for all $n \in P$ we have $\suc{n} \in P$.
%    Fix $n \in P$.
%    It suffices to show that for all $m'$ such that $m' \in \naturals$ we have for all $k' \in \naturals$ we have $\suc{n} \rmul (m' + k') = (\suc{n} \rmul m') + (\suc{n} \rmul k')$.
%    Fix $m' \in \naturals$.
%    Fix $k' \in \naturals$.
%    $n \in \naturals$.
%    $ \suc{n} \rmul (m' + k') = (n \rmul (m' + k')) + (m' + k') = ((n \rmul m') + (n \rmul k')) + (m' + k')  = ((n \rmul m') + (n \rmul k')) + m' + k'    = (((n \rmul m') + (n \rmul k')) + m') + k'  = (m' + ((n \rmul m') + (n \rmul k'))) + k'  = ((m' + (n \rmul m')) + (n \rmul k')) + k'  = (((n \rmul m') + m') + (n \rmul k')) + k'  = ((n \rmul m') + m') + ((n \rmul k') + k')  = (\suc{n} \rmul m') + (\suc{n} \rmul k')$.
%\end{proof}

\begin{proposition}\label{natural_disstro}
    Suppose $n,m,k \in \naturals$.
    Then $n \rmul (m + k) = (n \rmul m) + (n \rmul k)$.
\end{proposition}
\begin{proof}
   %Let $P = \{n \in \naturals \mid \forall m,k \in \naturals . n \rmul (m + k) = (n \rmul m) + (n \rmul k)\}$.
    Let $P = \{n \in \naturals \mid \text{for all $m,k \in \naturals$ we have $n \rmul (m + k) = (n \rmul m) + (n \rmul k)$}\}$.

    $\zero \in P$.
    $P \subseteq \naturals$.
    It suffices to show that for all $n \in P$ we have $\suc{n} \in P$.
    Fix $n \in P$.

    It suffices to show that for all $m'$ such that $m' \in \naturals$ we have for all $k' \in \naturals$ we have $\suc{n} \rmul (m' + k') = (\suc{n} \rmul m') + (\suc{n} \rmul k')$.
    Fix $m' \in \naturals$.
    Fix $k' \in \naturals$.
    $n \in \naturals$.
    \begin{align*}
        \suc{n} \rmul (m' + k') \\
        &= (n \rmul (m' + k')) + (m' + k') \\% \explanation{by \cref{naturals_mul_axiom_2}}\\
        &= ((n \rmul m') + (n \rmul k')) + (m' + k') \\%\explanation{by assumption}\\
        &= ((n \rmul m') + (n \rmul k')) + m' + k'   \\%\explanation{by \cref{naturals_add_remove_brakets,naturals_add_kommu}}\\
        &= (((n \rmul m') + (n \rmul k')) + m') + k' \\%\explanation{by \cref{naturals_add_kommu,naturals_add_remove_brakets,naturals_add_assoc}}\\
        &= (m' + ((n \rmul m') + (n \rmul k'))) + k' \\%\explanation{by \cref{naturals_add_kommu}}\\
        &= ((m' + (n \rmul m')) + (n \rmul k')) + k' \\%\explanation{by \cref{naturals_add_assoc}}\\
        &= (((n \rmul m') + m') + (n \rmul k')) + k' \\%\explanation{by \cref{naturals_add_kommu}}\\
        &= ((n \rmul m') + m') + ((n \rmul k') + k') \\%\explanation{by \cref{naturals_add_assoc}}\\
        &= (\suc{n} \rmul m') + (\suc{n} \rmul k')   %\explanation{by \cref{naturals_mul_axiom_2}}
    \end{align*}
\end{proof}

\begin{proposition}\label{naturals_rmul_one_kommu}
    For all $n \in \naturals$ we have $n \rmul 1 = n$.
\end{proposition}
\begin{proof}
    Let $P = \{ n \in \naturals \mid n \rmul 1 = n\}$.
    $1 \in P$.
    It suffices to show that for all $n' \in P$ we have $\suc{n'} \in P$.
    Fix $n' \in P$.
    It suffices to show that $\suc{n'} \rmul 1 = \suc{n'}$.
    \begin{align*}
        \suc{n'} \rmul 1 \\
        &=  n' \rmul 1 + 1 \\
        &=  n' + 1 \\
        &= \suc{n'}
    \end{align*}
\end{proof}

\begin{proposition}\label{naturals_rmul_kommu}
    Let $n, m \in \naturals$.
    Then $n \rmul m = m \rmul n$.
\end{proposition}
\begin{proof}
    Let $P = \{n \in \naturals \mid \forall m \in \naturals. n \rmul m = m \rmul n\}$.
    $\zero \in P$.
    $P \subseteq \naturals$.
    It suffices to show that for all $n \in P$ we have $\suc{n} \in P$.
    Fix $n \in P$.
    It suffices to show that for all $m \in \naturals$ we have $\suc{n} \rmul m = m \rmul \suc{n}$.
    Fix $m \in \naturals$.
    $n \rmul m = m \rmul n$.
    \begin{align*}
        \suc{n} \rmul m \\
        &= (n \rmul m) + m \\
        &= (m \rmul n) + m \\
        &= m + (m \rmul n) \\
        &= (m \rmul 1) + (m \rmul n) \\
        &= m \rmul (1 + n) \\
        &= m \rmul \suc{n} \\
    \end{align*}
\end{proof}

\begin{proposition}\label{naturals_rmul_assoc}
    Suppose $n,m,k \in \naturals$.
    Then $n \rmul (m \rmul k) = (n \rmul m) \rmul k$.
\end{proposition}
\begin{proof}
    Let $P = \{ n \in \naturals \mid \text{for all $m,k \in \naturals$ we have $n \rmul (m \rmul k) = (n \rmul m) \rmul k$ }\}$.
    $\zero \in P$.
    It suffices to show that for all $n' \in P$ we have $ \suc{n'} \in P$.
    Fix $n' \in P$.
    It suffices to show that for all $m' \in \naturals$ we have for all $k' \in \naturals$ we have $\suc{n'} \rmul (m' \rmul k') = (\suc{n'} \rmul m') \rmul k'$.
    Fix $m' \in \naturals$.
    Fix $k' \in \naturals$.
    \begin{align*}
        \suc{n'} \rmul (m' \rmul k')                \\
        &=(n' \rmul (m' \rmul k')) + (m' \rmul k')  \\
        &=((n' \rmul m') \rmul k') + (m' \rmul k')  \\
        &=(k' \rmul (n' \rmul m')) + (k' \rmul m')  \\
        &=k' \rmul ((n' \rmul m') + m')             \\
        &=k' \rmul (\suc{n'} \rmul m')              \\
        &=(\suc{n'} \rmul m') \rmul k'              
    \end{align*}
\end{proof}

\begin{lemma}\label{naturals_is_equal_to_two_times_naturals}
    $\{x+y \mid x \in \naturals, y \in \naturals \} = \naturals$.
\end{lemma}


\subsection{Axioms of the reals Part One}
We seen before that we can proof the common behavior of the naturals.
Since we now want to get furhter in a more efficient way, we introduce the basis axioms of the reals.
Such that we can introduce the integers and raitionals as smooth as possible.



The operations Multiplication and addition is closed in the reals
\begin{axiom}\label{reals_rmul}
    For all $n,m \in \reals$ we have $(n \rmul m) \in \reals$.
\end{axiom}

\begin{axiom}\label{reals_add}
    For all $n,m \in \reals$ we have $(n + m) \in \reals$.
\end{axiom}




Commutivatiy of the standart operations
\begin{axiom}\label{reals_axiom_kommu}
    For all $x,y \in \reals$ $x + y = y + x$ and $x \rmul y = y \rmul x$.
\end{axiom}
  
\begin{axiom}\label{reals_axiom_assoc}
    For all $x,y,z \in \reals$ we have $(x + y) + z = x + (y + z)$.
\end{axiom}


Existence of  one and Zero
\begin{axiom}\label{reals_axiom_zero}
    For all $x \in \reals$ $x + \zero = x$. 
\end{axiom}

\begin{axiom}\label{reals_axiom_one}
    For all $x \in \reals$ we have $x \rmul 1 = x$.
\end{axiom}



The Existence of Inverse of both operations
\begin{axiom}\label{reals_axiom_add_invers}
    For all $x \in \reals$ there exist $y \in \reals$ such that $x + y = \zero$.
\end{axiom}

\begin{axiom}\label{reals_axiom_mul_invers}
    For all $x \in \reals$ such that $x \neq \zero$ there exist $y \in \reals$ such that $x \rmul y = 1$.
\end{axiom}



Disstrubitiv Laws of the reals
\begin{axiom}\label{reals_axiom_disstro1}
    For all $x,y,z \in \reals$ $x \rmul (y + z) = (x \rmul y) + (x \rmul z)$.
\end{axiom}

\begin{axiom}\label{reals_disstro2}
    For all $x,y,z \in \reals$ $(y + z) \rmul x = (y \rmul x) + (z \rmul x)$.
\end{axiom}



Definition of the order symbols
\begin{abbreviation}\label{rless}
    $x < y$ iff $x \rless y$.
\end{abbreviation}

\begin{abbreviation}\label{less_on_reals}
    $x \leq y$ iff $x < y \lor x = y$.
\end{abbreviation}

\begin{abbreviation}\label{greater_on_reals}
    $x > y$ iff $y < x$.
\end{abbreviation}

\begin{abbreviation}\label{greatereq_on_reals}
    $x \geq y$ iff $y \leq x$.
\end{abbreviation}



Laws of the order on the reals
\begin{abbreviation}\label{is_positiv}
    $x$ is positiv iff $x > \zero$.
\end{abbreviation}

\begin{axiom}\label{reals_order}
    For all $x,y \in \reals$ we have if $x < y$ then $\lnot y < x$.
\end{axiom}

\begin{axiom}\label{reals_dense}
    For all $x,y \in \reals$ if $x < y$ then 
    there exist $z \in \reals$ such that $x < z$ and $z < y$.
\end{axiom}

\begin{axiom}\label{reals_is_eta_zero_set}
    For all $X,Y,x,y$ such that $X,Y \subseteq \reals$ and $x \in X$ and $y \in Y$ and $x < y$ we have there exist $z \in \reals$
    such that $x < z < y$.
\end{axiom}

\begin{axiom}\label{reals_one_bigger_zero}
    $\zero < 1$.
\end{axiom}

\begin{axiom}\label{reals_order_behavior_with_addition}
    For all $x,y \in \reals$ such that $x < y$ we have for all $z \in \reals$ $x + z < y + z$.
\end{axiom}

\begin{axiom}\label{reals_postiv_mul_is_positiv}
    For all $x,y \in \reals$ such that $\zero < x,y$ we have $\zero < (x \rmul y)$. 
\end{axiom}

\begin{axiom}\label{reals_postiv_mul_negativ_is_negativ}
    For all $x,y \in \reals$ such that $x < \zero < y$ we have $(x \rmul y) < \zero$. 
\end{axiom}

\begin{axiom}\label{reals_negativ_mul_is_negativ}
    For all $x,y \in \reals$ such that $x,y < \zero$ we have $\zero < (x \rmul y)$. 
\end{axiom}




\subsection{The Intergers}


\begin{axiom}\label{neg}
    For all $n \in \reals$ we have $\neg{n} \in \reals$ and $n + \neg{n} = \zero$.
\end{axiom}

\begin{axiom}\label{neg_of_zero}
    $\neg{\zero} = \zero$.
\end{axiom}

\begin{definition}\label{minus}
    $n - m = n + \neg{m}$.
\end{definition}

\begin{lemma}\label{minus_in_reals}
    Suppose $n,m \in \reals$.
    Then $n - m \in \reals$.
\end{lemma}
\begin{proof}
    \begin{byCase}
        \caseOf{$m = \zero$.} Trivial.
        \caseOf{$m \neq \zero$.} Trivial.
    \end{byCase}
\end{proof}

\begin{proposition}\label{negation_of_negation_is_id}
    For all $r \in \reals$ we have $\neg{\neg{r}} = r$.
\end{proposition}
\begin{proof}
    Omitted.
%    Fix $r \in \reals$.
%    $r + \neg{r} = \zero$.
%    $\neg{r} + \neg{\neg{r}} = \zero$.
%    Follows by \cref{neg,reals_axiom_kommu}.
\end{proof}

\begin{definition}\label{integers}
    $\integers = \{ z \in \reals \mid \exists n \in \naturals. z \in \naturals \lor  n + z = \zero\}$.
\end{definition}

\begin{lemma}\label{n_subset_z}
    $\naturals \subseteq \integers$.
\end{lemma}

\begin{lemma}\label{neg_of_naturals_in_integers}
    For all $n \in \naturals$ we have $\neg{n} \in \integers$.
\end{lemma}

\begin{lemma}\label{integers_eq_naturals_and_negativ_naturals}
    $\integers = \{ z \in \reals \mid \exists n \in \naturals. n = z \lor \neg{n} = z\}$.
\end{lemma}
\begin{proof}
    Omitted.
\end{proof}

\begin{abbreviation}\label{positiv_real_number}
    $r$ is a positiv real number iff $r > \zero$ and $r \in \reals$.
\end{abbreviation}



\subsection{The Rationals}

\begin{axiom}\label{invers_reals}
    For all $q \in \reals$ we have $\inv{q} \rmul q = 1$.
\end{axiom}

\begin{abbreviation}\label{rfrac}
    $\rfrac{x}{y} = x \rmul \inv{y}$.
\end{abbreviation}

\begin{abbreviation}\label{naturalsplus}
    $\naturalsPlus = \naturals \setminus \{\zero\}$.
\end{abbreviation}

\begin{definition}\label{rationals} %TODO: Vielleicht ist hier die definition das alle inversen von den ganzenzahlen und die ganzenzahlen selbst die rationalen zahlen erzeugen
    $\rationals = \{ q \in \reals \mid \exists z \in \integers. \exists n \in (\integers \setminus \{\zero\}). q = \rfrac{z}{n} \}$.
\end{definition}

\begin{abbreviation}\label{nominator}
    $z$ is nominator of $q$ iff there exists $n \in \naturalsPlus$ such that $q = \rfrac{z}{n}$.
\end{abbreviation}

\begin{abbreviation}\label{denominator}
    $n$ is denominator of $q$ iff there exists $z \in \integers$ such that $q = \rfrac{z}{n}$.
\end{abbreviation}

\begin{theorem}\label{one_divided_by_n_is_in_zero_to_one}
    For all $n \in \naturalsPlus$ we have $\zero < \rfrac{1}{n} \leq 1$.
\end{theorem}
\begin{proof}
    Omitted.
\end{proof}

\begin{lemma}\label{fraction_kuerzung} %TODO: Englischen namen rausfinden
    For all $n,m,k \in \reals$ we have $\rfrac{n \rmul k}{m \rmul k} = \rfrac{n}{m}$.
\end{lemma}
\begin{proof}
    Omitted.
\end{proof}

\begin{lemma}\label{fraction_swap}
    For all $n,m,k,l \in \reals$ we have $\rfrac{\rfrac{n}{m}}{\rfrac{k}{l}}=\rfrac{n \rmul l}{m \rmul k}$.
\end{lemma}
\begin{proof}
    Omitted.
\end{proof}

\begin{definition}\label{divisor}
    $g$ is a integral divisor of $a$ iff $g \in \naturals$ and there exist $b \in \integers$ such that $g \rmul b = a$.
\end{definition}




%\begin{abbreviation}\label{gcd}
%    $g$ is greatest common divisor of $\{a,b\}$ iff
%    $g$ is a integral divisor of $a$ 
%    and $g$ is a integral divisor of $b$ 
%    and for all $g'$ such that $g'$ is a integral divisor of $a$
%    and $g'$ is a integral divisor of $b$ we have $g' \leq g$.
%\end{abbreviation}
% TODO:  Was man noch so beweisen könnte: bruch kürzung, kehrbruch eigenschaften, bruch in bruch vereinfachung, 


\subsection{Order on the reals}

\begin{axiom}\label{reals_order_is_transitive}
    For all $x,y,z \in \reals$ such that $x < y$ and $y < z$ we have $x < z$.
\end{axiom}


\begin{lemma}\label{plus_one_order}
    For all $r\in \reals$ we have $ r < r + 1$.
\end{lemma}
\begin{proof}
    Omitted.
\end{proof}

\begin{lemma}\label{negation_and_order}
    Suppose $r \in \reals$.
    $r \leq \zero$ iff $\zero \leq \neg{r}$.
\end{lemma}
\begin{proof}
    Omitted.
\end{proof}


\begin{lemma}\label{reals_add_of_positiv}
    Let $x,y \in \reals$.
    Suppose $x$ is positiv and $y$ is positiv.
    Then $x + y$ is positiv.
\end{lemma}
\begin{proof}
    Omitted.
\end{proof}

\begin{lemma}\label{reals_mul_of_positiv}
    Let $x,y \in \reals$.
    Suppose $x$ is positiv and $y$ is positiv.
    Then $x \rmul y$ is positiv.
\end{lemma}
\begin{proof}
    Omitted.
\end{proof}



\begin{lemma}\label{order_reals_lemma0}
    For all $x \in \reals$ we have not $x < x$.
\end{lemma}
\begin{proof}
    Omitted.
\end{proof}


\begin{lemma}\label{order_reals_lemma1}
    Let $x,y,z \in \reals$.
    Suppose $\zero < x$. 
    Suppose $y < z$. 
    Then $(y \rmul x) < (z \rmul x)$.
\end{lemma}
\begin{proof}
    Omitted.
    %There exist $k \in \reals$ such that $y + k = z$ and $k > \zero$ by \cref{reals_definition_order_def}.
    %\begin{align*}
    %    (z \rmul x) \\
    %    &= ((y + k) \rmul x) \\
    %    &= ((y \rmul x) + (k \rmul x)) \explanation{by \cref{reals_disstro2}}
    %\end{align*} 
    %Then $(k \rmul x) > \zero$. 
    %Therefore $(z \rmul x) > (y \rmul x)$.
\end{proof}

\begin{lemma}\label{order_reals_lemma2}
    Let $x,y,z \in \reals$.
    Suppose $\zero < x$. 
    Suppose $y < z$. 
    Then $(x \rmul y) < (x \rmul z)$.
\end{lemma}
\begin{proof}
    Omitted.
\end{proof}


\begin{lemma}\label{order_reals_lemma3}
    Let $x,y,z \in \reals$.
    Suppose $\zero > x$. 
    Suppose $y < z$. 
    Then $(x \rmul z) < (x \rmul y)$.
\end{lemma}
\begin{proof}
    Omitted.
\end{proof}

\begin{lemma}\label{order_reals_lemma00}
    For all $x,y \in \reals$ such that $x > y$ we have $x \geq y$.
\end{lemma}
\begin{proof}
    Omitted.
\end{proof}

\begin{lemma}\label{order_reals_lemma5}
    For all $x,y \in \reals$ such that $x < y$ we have $x \leq y$.
\end{lemma}
\begin{proof}
    Omitted.
\end{proof}

\begin{lemma}\label{order_reals_lemma6}
    For all $x,y \in \reals$ such that $x \leq y \leq x$ we have $x=y$.
\end{lemma}
\begin{proof}
    Omitted.
\end{proof}

\begin{lemma}\label{reals_minus}
    Assume $x,y \in \reals$. If $x - y = \zero$ then $x=y$.
\end{lemma}
\begin{proof}
    Omitted.
\end{proof}

\begin{definition}\label{upper_bound}
    $x$ is an upper bound of $X$ iff for all $y \in X$ we have $x > y$.
\end{definition}

\begin{definition}\label{least_upper_bound}
    $x$ is a least upper bound of $X$ iff $x$ is an upper bound of $X$ and for all $y$ such that $y$ is an upper bound of $X$ we have $x \leq y$.
\end{definition}

\begin{lemma}\label{supremum_unique}
    %Let $x,y \in \reals$ and let $X$ be a subset of $\reals$.
    If $x$ is a least upper bound of $X$ and $y$ is a least upper bound of $X$ then $x = y$.
\end{lemma}
\begin{proof}
    Omitted.
\end{proof}

\begin{definition}\label{supremum_reals}
    $x$ is the supremum of $X$ iff $x$ is a least upper bound of $X$.
\end{definition}




\begin{definition}\label{lower_bound}
    $x$ is an lower bound of $X$ iff for all $y \in X$ we have $x < y$.
\end{definition}

\begin{definition}\label{greatest_lower_bound}
    $x$ is a greatest lower bound of $X$ iff $x$ is an lower bound of $X$ and for all $y$ such that $y$ is an lower bound of $X$ we have $x \geq y$.
\end{definition}

\begin{lemma}\label{infimum_unique}
    If $x$ is a greatest lower bound of $X$ and $y$ is a greatest lower bound of $X$ then $x = y$.
\end{lemma}
\begin{proof}
    Omitted.
\end{proof}

\begin{definition}\label{infimum_reals}
    $x$ is the supremum of $X$ iff $x$ is a greatest lower bound of $X$.
\end{definition}

\begin{definition}\label{minimum}
    $\min{X} = \{x \in X \mid \forall y \in X. x \leq y \}$.
\end{definition}


\begin{definition}\label{maximum}
    $\max{X} = \{x \in X \mid \forall y \in X. x \geq y \}$.
\end{definition}


\begin{definition}\label{intervalclosed}
    $\intervalclosed{a}{b} = \{x \in \reals \mid a \leq x \leq b\}$.
\end{definition}


\begin{definition}\label{intervalopen}
    $\intervalopen{a}{b} = \{ x \in \reals \mid a < x < b\}$.
\end{definition}

\begin{definition}\label{intervalopen_infinite_left}
    $\intervalopenInfiniteLeft{b} = \{ x \in \reals \mid x < b\}$.
\end{definition}

\begin{definition}\label{intervalopen_infinite_right}
    $\intervalopenInfiniteRight{a} = \{ x \in \reals \mid x > a\}$.
\end{definition}

\begin{definition}\label{intervalclosed_infinite_left}
    $\intervalclosedInfiniteLeft{b} = \{ x \in \reals \mid x \leq b\}$.
\end{definition}

\begin{definition}\label{intervalclosed_infinite_right}
    $\intervalclosedInfiniteRight{a} = \{ x \in \reals \mid x \geq a\}$.
\end{definition}


\begin{definition}\label{m_to_n_set}
    $\seq{m}{n} = \{x \in \naturals \mid  m \leq x \leq n\}$.   
\end{definition}

\begin{axiom}\label{abs_behavior1}
    If $x \geq \zero$ then $\abs{x} = x$.
\end{axiom}

\begin{axiom}\label{abs_behavior2}
    If $x < \zero$ then $\abs{x} = \neg{x}$.
\end{axiom}

\begin{definition}\label{realsminus}
    $\realsminus = \{r \in \reals \mid r < \zero\}$.
\end{definition}

\begin{definition}\label{realsplus}
    $\realsplus = \{r \in \reals \mid r > \zero\}$.
\end{definition}

\begin{definition}\label{epsilon_ball}
    $\epsBall{x}{\epsilon} = \intervalopen{x-\epsilon}{x+\epsilon}$.
\end{definition}








%\begin{proposition}\label{safe}
%    Contradiction.
%\end{proposition}
