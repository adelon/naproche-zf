\import{topology/topological-space.tex}


% T0 separation
\begin{definition}\label{is_kolmogorov}
    $X$ is Kolmogorov iff
    for all $x,y\in\carrier[X]$ such that $x\neq y$
    there exist $U\in\opens[X]$ such that
    $x\in U\not\ni y$ or $x\notin U\ni y$.
\end{definition}

\begin{abbreviation}\label{kolmogorov_space}
    $X$ is a Kolmogorov space iff $X$ is a topological space and
    $X$ is Kolmogorov.
\end{abbreviation}

\begin{abbreviation}\label{teezero}
    $X$ is \teezero\ iff $X$ is Kolmogorov.
\end{abbreviation}

\begin{abbreviation}\label{teezero_space}
    $X$ is a \teezero-space iff $X$ is a Kolmogorov space.
\end{abbreviation}

\begin{proposition}\label{kolmogorov_implies_kolmogorov_for_closeds}
    Suppose $X$ is a Kolmogorov space.
    Let $x,y\in\carrier[X]$.
    Suppose $x\neq y$.
    Then there exist $A\in\closeds{X}$ such that
    $x\in A\not\ni y$ or $x\notin A\ni y$.
\end{proposition}
\begin{proof}
    Take $U\in\opens[X]$ such that $x\in U\not\ni y$ or $x\notin U\ni y$
        by \cref{is_kolmogorov}.
    Then $\carrier[X]\setminus U\in\closeds{X}$ by \cref{complement_of_open_elem_closeds}.
    Now $x\in (\carrier[X]\setminus U)\not\ni y$ or $x\notin (\carrier[X]\setminus U)\ni y$
        by \cref{setminus}.
\end{proof}

\begin{proposition}\label{kolmogorov_for_closeds_implies_kolmogorov}
    Suppose for all $x,y\in\carrier[X]$ such that $x\neq y$
        there exist $U\in\closeds{X}$ such that
        $x\in U\not\ni y$ or $x\notin U\ni y$.
    Then $X$ is Kolmogorov.
\end{proposition}
\begin{proof}
    Follows by \cref{closeds,is_closed_in,is_kolmogorov,setminus}.
\end{proof}

\begin{proposition}\label{kolmogorov_iff_kolmogorov_for_closeds}
    Let $X$ be a topological space.
    $X$ is Kolmogorov iff
    for all $x,y\in\carrier[X]$ such that $x\neq y$
    there exist $U\in\closeds{X}$ such that
    $x\in U\not\ni y$ or $x\notin U\ni y$.
\end{proposition}
\begin{proof}
    Follows by \cref{kolmogorov_implies_kolmogorov_for_closeds,kolmogorov_for_closeds_implies_kolmogorov}.
\end{proof}

% T1 separation (Fréchet topology)
\begin{definition}\label{teeone}
    $X$ is \teeone\ iff
    for all $x,y\in\carrier[X]$ such that $x\neq y$
    there exist $U, V\in\opens[X]$ such that
    $U\ni x\notin V$ and $V\ni y\notin U$.
\end{definition}

\begin{abbreviation}\label{teeone_space}
    $X$ is a \teeone-space iff $X$ is a topological space and
    $X$ is \teeone.
\end{abbreviation}

\begin{proposition}\label{teeone_implies_singletons_closed}
    Let $X$ be a \teeone-space.
    Assume $x \in \carrier[X]$.
    Then $\{x\}$ is closed in $X$.
\end{proposition}
\begin{proof}
    Let $V = \{ U \in \opens[X] \mid x \notin U\}$.
    For all $y \in \carrier[X]$ such that $x \neq y$ there exist $U \in \opens[X]$ such that $x \notin U \ni y$.
    For all $y \in \carrier[X]$ such that $y \neq x$ there exists $U \in V$ such that $y \in U$.

    $\unions{V} \in \opens[X]$.
    For all $y \in \carrier[X]$ such that $x \neq y$ we have $y \in \unions{V}$.
    We show that $\carrier[X] \setminus \{x\} = \unions{V}$.
    \begin{subproof}
        We show that for all $y \in \carrier[X] \setminus \{x\}$ we have $y \in \unions{V}$.
        \begin{subproof}
            Fix $y \in \carrier[X] \setminus \{x\}$.
            $y \neq x$.
            $y \in \carrier[X]$.
            $y \in \unions{V}$.
        \end{subproof}
        For all $y \in \unions{V}$ we have $y \notin \{x\}$.
        For all $y \in \unions{V}$ we have $y \in \carrier[X] \setminus \{x\}$.
        Follows by set extensionality.
    \end{subproof}
\end{proof}
%
% Conversely, if \{x\} is open, then for any y distinct from x we can use
% X\setminus\{x\} as the open neighbourhood of y.

% T2 separation
\begin{definition}\label{is_hausdorff}
    $X$ is Hausdorff iff
    for all $x,y\in\carrier[X]$ such that $x\neq y$
    there exist $U, V\in\opens[X]$ such that
    $x\in U$ and $y\in V$ and $U$ is disjoint from $V$.
\end{definition}

\begin{abbreviation}\label{hausdorff_space}
    $X$ is a Hausdorff space iff $X$ is a topological space and
    $X$ is Hausdorff.
\end{abbreviation}

\begin{abbreviation}\label{teetwo}
    $X$ is \teetwo\ iff $X$ is Hausdorff.
\end{abbreviation}

\begin{abbreviation}\label{teetwo_space}
    $X$ is a \teetwo-space iff $X$ is a Hausdorff space.
\end{abbreviation}

\begin{proposition}\label{teeone_space_is_teezero_space}
    Let $X$ be a \teeone-space.
    Then $X$ is a \teezero-space.
\end{proposition}
\begin{proof}
    Follows by \cref{is_kolmogorov,teeone}.
\end{proof}

\begin{proposition}\label{teetwo_space_is_teeone_space}
    Let $X$ be a \teetwo-space.
    Then $X$ is a \teeone-space.
\end{proposition}
\begin{proof}
    We show that for all $x,y\in\carrier[X]$ such that $x\neq y$
    there exist $U, V\in\opens[X]$ such that
    $U\ni x\notin V$ and $V\ni y\notin U$.
    \begin{subproof}
        $X$ is hausdorff.
        For all $x,y\in\carrier[X]$ such that $x\neq y$
        there exist $U, V\in\opens[X]$ such that
        $x\in U$ and $y\in V$ and $U$ is disjoint from $V$.
    \end{subproof}
\end{proof}

\begin{definition}\label{regular}
    $X$ is regular iff for all $C,p$ such that $p \in \carrier[X]$ and $p \notin C \in \closeds{X}$ we have there exists $U,C \in \opens[X]$ such that $p \in U$ and $C \subseteq V$ and $U \inter V = \emptyset$.
\end{definition}

\begin{definition}\label{regular_space}
    $X$ is a regular space iff $X$ is a topological space and $X$ is regular.
\end{definition}


\begin{definition}\label{teethree}
    $X$ is \teethree\ iff $X$ is regular and $X$ is \teezero\ .
\end{definition}

\begin{definition}\label{teethree_space}
    $X$ is a \teethree-space iff $X$ is a topological space and $X$ is \teethree\ .
\end{definition}

\begin{proposition}\label{teethree_space_is_teetwo_space}
    Let $X$ be a \teethree-space.
    Then $X$ is a \teetwo-space.
\end{proposition}
\begin{proof}
    For all $x,y \in \carrier[X]$ such that $x \neq y$ we have $x \notin \{y\}$.
    It suffices to show that $X$ is hausdorff.
    It suffices to show that for all $x \in \carrier[X]$ we have for all $y \in \carrier[X]$ such that $y \neq x$ we have there exist $U,V \in \opens[X]$ such that $x\in U$ and $y \in V$ and $U$ is disjoint from $V$.
    Fix $x \in \carrier[X]$.
    It suffices to show that for all $y \in \carrier[X]$ such that $y \neq x$ we have there exist $U,V \in \opens[X]$ such that $x\in U$ and $y \in V$ and $U$ is disjoint from $V$.
    Fix $y \in \carrier[X]$.

    %There exist $U' \in \opens[X]$ such that $x\in U'\not\ni y$ or $x\notin U'\ni y$ by \cref{}.
    %There exist $C \in \closeds{X}$ such that $y \in C \not\ni X$.
    We show that there exist $U,V,C$ such that $U,V \in \opens[X]$ and $C\in \closeds{X}$ and $x \in U$ and $y \in C \subseteq V$ and $U$ is disjoint from $V$.
    \begin{subproof}
        Omitted.
    \end{subproof}
    $y \in V$.
    Follows by assumption.
\end{proof}

%    for all $x,y\in\carrier[X]$ such that $x\neq y$
%    there exist $U, V\in\opens[X]$ such that
%    $x\in U$ and $y\in V$ and $U$ is disjoint from $V$.