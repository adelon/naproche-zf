\import{set.tex}
\import{relation.tex}

\subsection{Properties of relations}

\begin{definition}\label{left_quasireflexive}
    $R$ is left quasireflexive iff
    for all $x, y$ such that $x\mathrel{R} y$
    we have $x\mathrel{R}x$.
\end{definition}

\begin{definition}\label{right_quasireflexive}
    $R$ is right quasireflexive iff
    for all $x, y$ such that $x\mathrel{R} y$
    we have $y\mathrel{R}y$.
\end{definition}

\begin{definition}\label{quasireflexive}
    $R$ is quasireflexive iff
    for all $x, y$ such that $x\mathrel{R} y$
    we have $x\mathrel{R}x$ and $y\mathrel{R}y$.
\end{definition}

\begin{definition}\label{coreflexive}
    $R$ is coreflexive iff
    for all $x, y$ such that $x\mathrel{R} y$
    we have $x = y$.
\end{definition}

\begin{definition}\label{reflexive_on}
    $R$ is reflexive on $X$ iff
    for all $x\in X$
    we have $x\mathrel{R}x$.
\end{definition}

\begin{definition}\label{irreflexive}
    $R$ is irreflexive iff
    for all $x$ we have $(x,x)\notin R$.
\end{definition}


\begin{proposition}\label{quasireflexive_implies_reflexive_on_fld}
    Suppose $R$ is quasireflexive.
    Then $R$ is reflexive on $\fld{R}$.
\end{proposition}

\begin{proposition}\label{reflexive_on_fld_impliesquasireflexive}
    Suppose $R$ is reflexive on $\fld{R}$.
    Then $R$ is quasireflexive.
\end{proposition}

\begin{proposition}\label{inters_of_family_of_reflexive_relations_is_reflexive}
    Let $F$ be an inhabited family of relations.
    Suppose every element of $F$ is reflexive on $A$.
    Then $\inters{F}$ is reflexive on $A$.
\end{proposition}
\begin{proof}
    For all $a\in A$ we have for all $R\in F$ we have $a\mathrel{R} a$.
    Thus for all $a\in A$ we have $a\mathrel{(\inters{F})} a$.
\end{proof}

\begin{definition}\label{antisymmetric}
    $R$ is antisymmetric iff
    for all $x, y$ such that $x\mathrel{R}y\mathrel{R}x$
    we have $x = y$.
\end{definition}

\begin{definition}[Symmetry]\label{symmetric}
    $R$ is symmetric iff
    for all $x, y$ we have
    $x\mathrel{R} y \iff y\mathrel{R}x$.
\end{definition}

\begin{definition}\label{asymmetric}
    $R$ is asymmetric iff
    for all $x, y$ such that
    $x\mathrel{R} y$ we have $y\not\mathrel{R}x$.
\end{definition}

\begin{proposition}\label{asymmetric_implies_irreflexive}
    Suppose $R$ is asymmetric.
    Then $R$ is irreflexive.
\end{proposition}

\begin{proposition}\label{asymmetric_implies_antisymmetric}
    Suppose $R$ is asymmetric.
    Then $R$ is antisymmetric.
\end{proposition}

\begin{proposition}\label{antisymmetric_and_irreflexive_implies_asymmetric}
    Suppose $R$ is antisymmetric.
    Suppose $R$ is irreflexive.
    Then $R$ is asymmetric.
\end{proposition}

% TODO
%\begin{proposition}\label{symmetric_relation_eq_converse}
%    Let $R$ be a symmetric relation.
%    Then $\converse{R} = R$.
%\end{proposition}
%\begin{proof}
%    Follows by set extensionality.
%\end{proof}

\begin{definition}[Transitivity]\label{transitive}
    $R$ is transitive iff
    for all $x, y, z$ such that $x\mathrel{R}y\mathrel{R}z$
        we have $x\mathrel{R} z$.
\end{definition}

\begin{proposition}\label{transitive_downward_elem}
    Suppose $R$ is transitive.
    Suppose $a\in\downward{R}{b}$.
    Suppose $c\in\downward{R}{a}$.
    Then $c\in \downward{R}{b}$.
\end{proposition}
\begin{proof}
    $c\mathrel{R} a\mathrel{R}b$.
    Thus $c\mathrel{R} b$ by \hyperref[transitive]{transitivity}.
\end{proof}

\begin{proposition}\label{transitive_downward_subseteq}
    Suppose $R$ is transitive.
    Suppose $a\in\downward{R}{b}$.
    Then $\downward{R}{a}\subseteq \downward{R}{b}$.
\end{proposition}

\begin{definition}\label{dense}
    $R$ is dense iff
    for all $x, z$ such that $x\mathrel{R}z$
        there exists $y$ such that $x\mathrel{R}y\mathrel{R}z$.
\end{definition}

% Variation of connexity that does not depend on a carrier.
\begin{definition}\label{quasiconnex}
    $R$ is quasiconnex iff
    for all $x, y\in\fld{R}$ such that $x\neq y$ we have
        $x\mathrel{R}y$ or $y\mathrel{R}x$.
\end{definition}

% Also called "connected" (here reserved for topology),
% "complete" (heavily overloaded), and total" (also overloaded).
\begin{definition}\label{connex}
    $R$ is connex on $X$ iff
    for all $x, y\in X$ such that $x\neq y$ we have
        $x\mathrel{R}y$ or $y\mathrel{R}x$.
\end{definition}

\begin{definition}\label{strongly_quasiconnex}
    $R$ is strongly quasiconnex iff
    for all $x, y\in\fld{R}$ we have
        $x\mathrel{R}y$ or $y\mathrel{R}x$.
\end{definition}

\begin{definition}\label{strongly_connex}
    $R$ is strongly connex on $X$ iff
    for all $x, y\in X$ we have
        $x\mathrel{R}y$ or $y\mathrel{R}x$.
\end{definition}

%\begin{proposition}\label{strongly_quasiconnex_implies_quasiconnex_and_quasireflexive}
%    Suppose $R$ is strongly quasiconnex.
%    Then $R$ is quasiconnex and quasireflexive.
%\end{proposition}

%\begin{proposition}\label{quasiconnex_and_quasireflexive_implies_strongly_quasiconnex}
%    Suppose $R$ is quasiconnex and quasireflexive.
%    Then $R$ is strongly quasiconnex.
%\end{proposition}

\begin{proposition}\label{strongly_quasiconnex_iff_quasiconnex_and_quasireflexive}
    $R$ is strongly quasiconnex iff
    $R$ is quasiconnex and quasireflexive.
\end{proposition}
\begin{proof}
    Follows by \cref{quasiconnex,strongly_quasiconnex,quasireflexive_implies_reflexive_on_fld,reflexive_on,fld,reflexive_on_fld_impliesquasireflexive}.
\end{proof}

\begin{proposition}\label{connex_reaches_all_or_all_but_one}
    Suppose $R$ is connex on $A$.
    Let $a, b\in A\setminus\ran{R}$.
    Then $a = b$.
\end{proposition}
\begin{proof}
    Suppose not.
    $a, b\in A$.
    Then $(a, b)\in R$ or $(b,a)\in R$ by \cref{connex}.
    $(a, b)\notin R$.
    $(b, a)\notin R$.
    Thus $a = b$.
\end{proof}


\begin{definition}\label{righteuclidean}
    $R$ is right Euclidean iff
    for all $a,b,c$ such that $a\mathrel{R}b,c$ we have $b\mathrel{R}c$.
\end{definition}

\begin{definition}\label{lefteuclidean}
    $R$ is left Euclidean iff
    for all $a,b,c$ such that $a,b\mathrel{R}c$ we have $a\mathrel{R}b$.
\end{definition}
