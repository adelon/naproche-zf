\import{relation.tex}
\import{relation/properties.tex}

\subsection{Quasiorders}

% also called preorder
\begin{abbreviation}\label{quasiorder}
    $R$ is a quasiorder iff
    $R$ is quasireflexive and transitive.
\end{abbreviation}

% also called preorder
\begin{abbreviation}\label{quasiorder_on}
    $R$ is a quasiorder on $A$ iff
        $R$ is a binary relation on $A$ and
        $R$ is reflexive on $A$ and transitive.
\end{abbreviation}

\begin{struct}\label{quasiordered_set}
    A quasiordered set $X$ is a onesorted structure
    equipped with
        \begin{enumerate}
            \item $\lt$
        \end{enumerate}
    such that
    \begin{enumerate}
        \item\label{quasiorder_type} $\lt[X]$ is a binary relation on $\carrier[X]$.
        \item\label{quasiorder_refl} $\lt[X]$ is reflexive on $\carrier[X]$.
        \item\label{quasiorder_tran} $\lt[X]$ is transitive.
    \end{enumerate}
\end{struct}

\begin{lemma}\label{quasiorder_transitive_double}
    Let $X$ be a quasiordered set.
    Let $a, b, c, d \in X$.
    Suppose $a\mathrel{\lt[X]} b\mathrel{\lt[X]} c\mathrel{\lt[X]} d$.
    Then $a\mathrel{\lt[X]} d$.
\end{lemma}
\begin{proof}
    $\lt[X]$ is transitive.
    Thus $a\mathrel{\lt[X]} c\mathrel{\lt[X]} d$ by \hyperref[transitive]{transitivity}.
    Hence $a\mathrel{\lt[X]} d$ by \hyperref[transitive]{transitivity}.
\end{proof}

\begin{proposition}\label{subseteqrel_is_quasiorder}
    $\subseteqrel{A}$ is a quasiorder on $A$.
\end{proposition}
\begin{proof}
    $\subseteqrel{A}$ is reflexive on $A$.
    $\subseteqrel{A}$ is transitive.
\end{proof}
